\section{Área de estudio (Unidad de análisis)}
Se definió la unidad de análisis en la costa norte y centro del Perú, considerada
como una zona altamente productiva para el cultivo de caña de azúcar debido a sus condiciones naturales que incluyen: temperatura con 
oscilaciones moderadas, disponibilidad constante de agua para riego y suelos fértiles de origen aluvial 
\citep{helfgott_cultivo_2016}.  

\begin{figure}[H]
    \centering
    \caption{Ubicación de los principales ingenios azucareros del país que conforman el gremio PERUCAÑA.}
    \includegraphics[width=0.6\textwidth]{img/6_metodologia/perucaña.png}
    \label{fig:ubicacion_cana}
    \begin{flushleft}
        \textit{Nota.} El ingenio de Caña Brava es administrado por la empresa Agrícola del Chira S.A. Elaboración propia con datos de \href{https://youtu.be/CMTbgXX8cGA?t=85}{PERUCAÑA}.     
        \vspace{-\baselineskip}
    \end{flushleft}
\end{figure}

La industria azucarera se concentra en cinco regiones: Piura, Lambayeque, La Libertad, Áncash y Lima (Figura \ref{fig:ubicacion_cana}).
De estas regiones, La Libertad destaca en términos de superficie cultivada con un 47\% del total.  Asímismo, las ocho empresas azucareras asociadas al gremio exportan más del 97 \% del volumen 
comercializado a nivel nacional, según datos de \citet{perucana_industria_2022}.



