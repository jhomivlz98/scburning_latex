\section{Planteamiento del problema}
\subsection{Descripción de la problemática}
%% Poner dos casos a nivel mundial donde se relacione la quema con las enfermedades, comenzar a mencionar por países.
La caña de azúcar es el cultivo azucarero más importante a nivel mundial con presencia en ciento dieciséis países tropicales y subtropicales, se destaca por ser la principal fuente de azúcar 
y la segunda fuente de etanol después del maíz \citep{oecd-fao_2023}. De entre esos países, los mayores productores son Brasil, India, China, 
Tailandia y Estados Unidos, quienes representan juntos más del 70\% de la producción mundial \citep{fao_2023}. 

Para su proceso de producción se utiliza entre el 75 y 80\% de la planta de caña de azúcar principalmente los tallos, mientras que el restante es el residuo compuesto
de tallos secos, cogollos, raíces, hojas verdes y secas. Este residuo no es aprovechable en la industria y una práctica muy extendida a nivel mundial para su eliminación es la quema \citep{singh_impact_2018}. 

\citet{franca_pre-harvest_2012} menciona que la quema de los residuos de caña de azúcar genera un aumento significativo de contaminantes hacia la atmósfera como óxidos de carbono ($CO_{2}$ y $CO$), óxidos de nitrógeno ($NO_{x}$), hidrocarburos ($HC$), 
material particulado fino ($PM_{2.5}$) y grueso ($PM_{10}$).  Estos contaminantes tienen efectos adversos en la calidad del aire y representan un riesgo significativo para la 
salud pública y el medio ambiente. 

En la ciudad de Araquara en Brasil, los eventos de quema de caña de azúcar en los períodos de cosecha 2002-2004 generaron una concentración de $PM_{10}$ 
tres veces mayor y una cantidad de hidrocarburos aromáticos políciclicos (PAH) cuatro veces mayor en comparación a los períodos sin quema \citep{de_andrade_contribution_2010}. De manera similar, \citet{junpen_estimation_2020}  
reportaron que en la región norte de Tailandia durante el período de cosecha 2018-2019, en los días que se llevó a cabo la quema de caña de azúcar, la concentración mensual promedio de $PM_{2.5}$ superó por más del doble los 
valores promedios anuales límites recomendados por la \citet{oms_2021} de 10 $\mu g/m^3$.

En Piraicaba, Brasil, durante el período de cosecha de caña 1997-1998 se asoció
el incremento de la concentración de $PM_{2.5}$ y $PM_{10}$, de hasta $10.2 \mu g/m^3$ y $42.9 \mu g/m^3$ respectivamente, con un aumento del 21.4\% en el número de niños y del 31.03\% en ancianos diagnosticados con enfermedades respiratorias \citep{cancado_impact_2006}.
Posteriormente,  \citet{arbex_air_2007} relacionó directamente el aumento de las emisiones de material particulado durante los períodos de quema de caña de azúcar con un incremento del 11.6\% en las admisiones hospitalarias por asma 
en un hospital de la ciudad de Araraquara, Brasil.

\citet{silveira_emissions_2013} observaron que los hidrocarburos aromáticos policíclicos (PAH), resultantes de la quema, tienen una asociación directa con efectos genotóxicos en trabajadores de caña en la ciudad brasileña de Barretos, lo que sugiere un incremento en los casos 
de riesgo de cáncer en las personas afectadas por las emisiones. \citet{pestana_effects_2017} también asoció 
el aumento de los niveles de $NO_{2}$ con un incremento del 1.12\% en las admisiones hospitalarias por enfermedades cardiovasculares en la región del oeste de São Paulo.

\citet{nowell_impacts_2022} estimaron una mortalidad de 2.5 muertes por año en veinte condados del sur de Florida, en Estados Unidos, debido a la emisión de $PM_{2.5}$ producida por los incendios de caña de azúcar, las causas de las muertes estaban 
relacionadas con vías respiratorias, enfermedades pulmonares y cáncer de pulmón. Asímismo, se ha comprobado que la quema de caña genera síntomas inflamatorios oculares 
y afecta la calidad de la mucosidad nasal \citep{matsuda_occupational_2020}. 

En la tesis de \citet[p. 42]{salas_quema_2010} se menciona que la quema de caña de azúcar realizada antes de la cosecha afecta a la biodiversidad del suelo, provocando la pérdida de nitrógeno, la población de microorganismos del suelo y el material orgánico disponible en el mismo.

%% Contexto nacional
En el Perú esta actividad, ha generado numerosas denuncias por parte de la población de las regiones de  Piura, La Libertad, Lambayeque, Ancash y
Lima debido a su impacto negativo en la salud y la calidad del aire. Estas denuncias reportadas por diversos medios de comunicación se encuentran documentadas en el proyecto de ley N° 06243/2020-CR, compilado por el \citet[pp. 12-25]{congreso_de_la_republica_ley_2020}. 

Dentro de las instituciones y actores relevantes a nivel nacional, el Organismo de Evaluación y Fiscalización Ambiental (OEFA) es responsable de supervisar la quema controlada de la caña de azúcar en las empresas azucareras, estableciendo restricciones basadas en variables 
como distancia a centro poblados, velocidad - dirección del viento, horario, duración y extensión del área de quema por día \citep{oefa_canazucar_2023}. Estas medidas representan un primer paso hacia la implementación de la cosecha en verde como reemplazo gradual de la quema
\citep{agrolmos_sa_modificacion_2021,agroaurora_sac_actualizacion_2020}.

A pesar de estos esfuerzos, hasta la fecha solo seis empresas azucareras, de las veintiún administradas por el OEFA, han adoptado dentro de sus instrumentos
de gestión ambiental (IGA) un compromiso y planificación para la implementación de la cosecha mecanizada en verde, es decir sin quema. Debido a que el cumplimiento de estos compromisos depende en gran medida de la información 
proporcionada por las empresas, es considerada insuficiente en la mayoría de casos, no controla la quema provocada por terceros y es ineficaz en las empresas azucareras que no cuentan con un instrumento de gestión ambiental; aquello representa
un problema relevante para la identificación y control de la quema en los lugares donde ocurre \citep[pp. 1530 - 1536]{casa_grande_saa_2018}

En este sentido, se resalta la necesidad de desarrollar un modelo de detección de quema de caña de azúcar en el Perú que permita brindar información precisa y oportuna a las autoridades pertinentes sobre la identificación y extensión de las áreas 
con quema de caña de azúcar. Esto permitirá intensificar las labores de monitoreo y fiscalización en dichas zonas.

\subsection{Formulación del problema}
\subsubsection{Pregunta general.}
¿Es posible desarrollar un modelo de aprendizaje ensamblado para la detección de quema de caña de azúcar?
\subsubsection{Preguntas específicas.}
\begin{itemize}
    \item ¿Cómo generar el conjunto de datos necesario para el funcionamiento del modelo de aprendizaje ensamblado para detectar la quema de caña de azúcar?
    \item ¿Cuáles son los algoritmos que mejor se ensamblan al modelo para detectar la quema de caña de azúcar?
    \item ¿El modelo de aprendizaje ensamblado presenta un desempeño superior en comparación con los algoritmos individuales?
\end{itemize}